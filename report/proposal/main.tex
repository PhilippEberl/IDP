\documentclass[11pt, oneside]{article}
\usepackage[utf8]{inputenc}
\usepackage{graphicx}
\usepackage{subfiles}
\usepackage{geometry}
\usepackage{multirow}
\usepackage[parfill]{parskip}
\usepackage{float}
\geometry{letterpaper}


\begin{document}
\subfile{sections/titlepage.tex}

\section{Abstract}

This project aims to leverage knowledge graph construction techniques to comprehensively analyze the UK government's dataset on traffic accidents. By integrating advanced data science methodologies, the project seeks to extract valuable insights, identify key patterns, and construct a knowledge graph to enhance the understanding of the complex factors contributing to traffic accidents in the UK.

\section{Research question}

What are the significant factors contributing to the occurrence and severity of traffic accidents in the UK?

How can knowledge graph representation facilitate the identification of critical relationships and patterns within the UK traffic accidents dataset, enabling the development of targeted interventions for accident prevention?

\section{Methodology}

Specific techniques from the Data Science Curriculum, including graph theory for knowledge graph construction, and statistical analysis for EDA, will be applied. Emphasis will be placed on ensuring the reproducibility of results, maintaining a clear and documented workflow, and adhering to ethical guidelines in data handling and analysis.

\section{Expected results}

The project anticipates the identification of key semantic features related to accident causality, including critical weather conditions, high-risk geographical locations, and specific vehicle manoeuvres leading to severe casualties. Additionally, the construction of a comprehensive knowledge graph is expected to reveal intricate relationships and dependencies among various accident-related attributes, providing a holistic understanding of the underlying factors contributing to traffic accidents in the UK. The project aims to generate actionable insights that can guide the implementation of targeted road safety measures and policies, ultimately contributing to a reduction in the frequency and severity of traffic accidents across the country.

\end{document}